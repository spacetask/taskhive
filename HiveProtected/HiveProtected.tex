\documentclass[a4paper,12pt]{article}

\renewcommand{\familydefault}{\sfdefault}
\usepackage[australian]{babel}
\usepackage{enumitem}
\usepackage[margin=1in]{geometry}
\usepackage{graphicx}
\usepackage{xcolor}
	\definecolor{branding}{HTML}{FFD600}
\usepackage{parskip}
\setlength{\parskip}{1em}
\usepackage[framemethod=tikz]{mdframed}
	\newmdenv[%
		innerlinewidth=6pt,%
		roundcorner=4pt,%
		linecolor=black,%
		backgroundcolor=branding,%
		innerleftmargin=2em,%
		innerrightmargin=2em,%
		innertopmargin=2em,%
		innerbottommargin=2em,
		nobreak=true]%
		{titlebox}
	\newmdenv[%
		innerlinewidth=0.5pt,%
		linecolor=black,%
		innerleftmargin=0.5em,%
		innerrightmargin=0.5em,%
		innertopmargin=0.5em,%
		innerbottommargin=0.5em,%
		skipabove=1em,%
		nobreak=true]%
		{optionalbox}

\setcounter{secnumdepth}{5}

\usepackage{latexsym, textcomp}
\usepackage{framed}
\usepackage{tabularx}
\usepackage{draftwatermark}
	\SetWatermarkScale{5}

\renewcommand{\labelenumi}{(\arabic{enumi})}
\renewcommand{\labelenumii}{(\alph{enumii})}
\renewcommand{\labelenumiii}{(\roman{enumiii})}
\renewcommand{\labelenumiv}{(\Alph{enumiv})}

\interfootnotelinepenalty=10000

\begin{document}

\vspace*{\fill}

\begin{titlebox}
\begin{center}
\Huge{\textbf{HiveProtected Contracts}}

\vspace{1em}

\large{Challenges, Opportunities and Solutions for Creating and Enforcing Pseudonymous Contracts}

\vspace{1em}

\large{\today}
\end{center}
\end{titlebox}

\vfill

\begin{center}
\includegraphics[width=0.5\textwidth]{TH.png}\\
\includegraphics[width=0.5\textwidth]{EH.png}
\end{center}

\newpage

\setcounter{tocdepth}{2}

\tableofcontents

\section{What Is an HiveProtected Contract?}

HiveProtected contracts (`HPCs') take advantage of Taskhive's security features to create agreements that are easily understood and enforced. The realisation of HPCs is based on two design principles:

\begin{itemize}
	\item Modularity --- each component is intended to be discrete, allowing for future changes to be made to each module without significantly affecting others. This approach is feasible, as contracts generally tend to be quite modular, consisting of groups of clauses which can to a large extent be modified independently.
	\item Pragmatism --- although Taskhive is intended to embody free market principles, some restrictions are necessary due to practical and technical constraints, and to ensure a useable system can be launched. In accordance with the modular approach, new features can be easily added over time.
\end{itemize}

This document sets out the challenges HPCs face compared to traditional contracts, and the opportunities created through innovative solutions to those challenges. Immediate and long-term features are discussed and recommended. Future documents will follow post-launch, outlining new features that could be introduced to better realise Taskhive's goals. For now, however, this document suggests a basic set of features for a functional contract-making system.

\section{Forming a Contract}

In legal systems descended from and including English law (such as American, Australian and Canadian law, among others) contracts generally begin with an offer-acceptance pair. One person (`the offeror') will make an offer to form an agreement on particular terms, and another person to whom the offer was made (`the offeree') may accept those terms. The combination of offer and acceptance indicates that the parties are `of one mind' as to the terms of the agreement. In essence, an offer must be able to be accepted `as-is', and no further negotiation would be required to form a contract. If an offeree wants to modify the offered terms before accepting, they are in fact making a counter-offer and the roles are reversed (the original offeror is now the offeree, and vice versa).

An offer may be made in several ways, including:

\begin{itemize}
	\item an offer made `to the world' which anyone may accept (but the form of acceptance may be specified),
	\item an offer made to a limited class or number of classes of people (such as residents with a particular postcode), or
	\item an offer made only to a specific person.
\end{itemize}

Sometimes a person will not be making either an offer or a counter-offer, but simply be indicating a willingness to negotiate a contract with a view of eventually offering or accepting terms. This is called an `invitation to treat', and is not binding either when made or accepted. An invitation to treat allows a person to express their interest in forming an agreement without being bound. Offers, on the other hand, are binding and able to be accepted until they are revoked, and offers are revocable until they are accepted or otherwise expire in accordance with the offer's terms. In Taskhive, an offer (in the contractual sense) should be considered made when a one user signs an agreement (becoming the offering party), and an acceptance when the second user signs the agreement (the offer is binding as all it requires is acceptance). If an offeree modifies the agreement they have been offered and signs it, this should be considered a counter-offer, which requires the original offeror to accept.

In the long term, Taskhive should develop to cover all eventualities, including offers which can only be accepted, offers with a willingness to consider counter-offers, invitations to treat which invite the making of offers, and invitations to treat which merely indicate willingness to begin negotiations; as well as giving users a choice of making offers and invitations to treat to all users, users with particular attributes (such as a minimum reputation score), or all users. However, this is not essential for the initial release of Taskhive. Initially it will be sufficient to consider all listings as a starting point for negotiations which will eventually lead to a binding offer and acceptance.

\subsection{Recommended Short-term Approach}

Initially a HPC should be created in the following manner:

\begin{enumerate}
	\item All task postings should be considered \textit{invitations to treat}.
	\item Task-specific addresses should be used to negotiate the terms of the agreement.
	\item Once the terms have been agreed, one party should prepare, review and sign the agreement; this should be considered an offer if it is automatically sent to the other user.
	\item The offer should be binding until revoked or it expires; revocation should be permitted until the offer is accepted.
	\item If the user agrees with the terms, they should accept the offer by signing and the contract should become binding.
	\item If the user disagrees with the terms, they should not sign, and negotiations should continue.
\end{enumerate}

\subsection{Recommended Long-term Approach}

Although it may seem desirable according to Taskhive's marketplace ethos to require all contracts to be negotiated, contractual freedom is a key aspect of a free market. It may be appealing to force negotiation (or at least the consideration of counter-offers on the basis that it would encourage compromise and lead to more contracts being successfully formed, this is paternalistic and contrary to the notion that individuals should be free to make their own economic decisions. The conclusion is that, over time, Taskhive should expand, and, in the long-term:

\begin{enumerate}
	\item A user posting a task should be able to choose between making:
	\begin{enumerate}
		\item an offer that can be accepted immediately,
		\item an offer indicating a willingness to consider counter-offers,
		\item an invitation to treat that invites the making of offers, or
		\item an invitation to treat that invites negotiations.
	\end{enumerate}
	\item A user posting a task should be able to choose between making an offer or invitation to treat to:
	\begin{enumerate}
		\item all users,
		\item some users (restricted to a class or several classes of user), or
		\item to a specific user.
	\end{enumerate}
	\item A user who receives a signed agreement (an offer) should be able to:
	\begin{enumerate}
		\item accept it by signing, or
		\item modify and sign it, making a counter-offer which is returned to the original offeror.
	\end{enumerate}
\end{enumerate}

\section{Overview of Modules}

HPCs will consist of several modules, each being a separate section of the agreement dealing with a specific issue, and will consist of clauses and subclauses, some of which will be mandatory, while others may be optional, contain user-editable information or require user input. Below is a list of these modules, their abbreviations and a brief description.

\begin{description}
	\item[Parties] identifies the parties and lists their communication addresses.
	\item[Interpretation] defines key terms and provides that the agreement is to be interpreted according to the definitions and interpretation rules.
	\item[Dispute Resolution] specifies that disputes are to be resolved by arbitration in accordance with the arbitration rules, identifies the arbitrator and lists their communication address and fees, and specifies the language of the proceedings.
	\item[Primary Obligations] specifies the main obligations each party assumes.
	\item[Intellectual Property] specifies the arrangements for ownership, licensing and assignment of intellectual property involved in the agreement.
	\item[Termination \& Penalties] provides for the early discontinuation of the agreement, including penalties and forfeits parties may be required to pay.
	\item[Escrow] specifies the amounts each party is required to pay into escrow, and provides a right of termination where payments are not promptly made after execution of the agreement.
	\item[Confidentiality] allows parties to choose confidentiality (non-disclosure) requirements.
\end{description}

\section{Parties}

Traditionally it has been essential to the enforceability of a contract that its parties are identified with precision. Without such, the parties would have difficulty proving either that they are entitled to enforce the agreement or against whom they are entitled to enforce it, and be unable to serve legal documents. Details typically include:

\begin{itemize}
	\item for individuals --- their full name, trading name and business number (if applicable), and business address (and, in some cases, identity information such as a driver's licence or passport number), or
	\item for companies --- their full legal name and form, registration number, place of incorporation, and registered office address.
\end{itemize}

These details form a necessary interface between the parties and the court or tribunal adjudicating a contractual. In Taskhive, however, each user has a unique public key (which for all intents and purposes is analogous to a party's name) and each task has a set of unique electronic addresses by which users may communicate. While an individual or company may be identifiable from this if they, for example, list their public key on their website, this cannot be guaranteed because it would be contrary to Taskhive's basis of facilitating pseudonymity. As a result, public keys are generally useless to courts.

Despite this, actions performed by a user within the Taskhive-HPC system can undeniably linked to that user's public key. Consequently, the public key will be sufficient to enforce the agreement when backed up by the mandatory escrow and arbitration provisions outlined below. While HPCs will not be enforceable in a court, they will be practically enforceable via the dispute resolution and anti-fraud measures of the Taskhive-HPC system.

It is recommended that each party be described as `the Vendor' or `the Vendee' to avoid confusion. A public key is not particularly human-readable, so inserting each party's public key whenever they are referred to may lead to errors, especially on the part of the arbitrator. If a user is providing a service they should be identified throughout the document as `the Vendor'; the other party should be identified throughout the document as `the Vendee'.

The proposed Parties module (\texttt{MODULE P}) would list the parties' public keys and task-specific addresses, and assign them human-readable names: either `the Vendor' or `the Vendee' depending on their role in the contract. The Vendor is the person providing a good or service, and the Vendee is the person receiving the good or service. In common terminology these might be called `seller' and `buyer' or `service provider' and `client' respectively, however `vendor' and `vendee' are broader terms for the parties in these sorts of relationships.

\section{Interpretation}

The proposed Interpretation module (\texttt{MODULE I}) defines key terms used frequently in the agreement, and is intended not to be editable by users. It also makes reference to the Interpretation Rules module (\texttt{MODULE IR}) which sets out the rules which parties and arbitrators are to use to interpret the agreement (this is covered in more detail below).

\subsection{Definitions}

A definitions clause is commonly included in contracts to ensure that words and phrases used in the agreement are interpreted consistently, rather than relying on context to determine the intended meaning. This is especially important where an ambiguous word is used, or where a word is used in an unusual sense. A definitions clause can be useful to refer readers to other relevant clauses if they are looking for how a word is defined.

The proposed Interpretation module (\texttt{MODULE I}) would contain a fixed definitions clause to ensure that users with limited knowledge or understanding cannot erroneously change the effect of clauses dependent on those definitions. There are likely to be only a few terms which need to be defined in the definitions clause.

\subsection{Governing Law}

A governing law clause specifies the laws in accordance with which a contract is made: that is, which law applies when enforcing and interpreting the agreement. It does not specify in which jurisdiction or court or tribunal a dispute is to be adjudicated. Governing law clauses can affect both the form and substance of a contract, and determine to a large extent how a contract is to be interpreted.

In English law and many other common law legal systems, for example, contracts for the sale of land must be in writing and signed by both parties. This is an example of the governing law affecting the form of an agreement. Contracts may also be found void for illegality if the subject of the transaction is prohibited (such as a contract for murder or sale of illegal drugs). Each legal system also has rules governing the interpretation of contracts when a court or arbitrator is adjudicating dispute.

In the absence of a governing law clause, the laws of a relevant jurisdiction will typically apply. `Relevant' may be the jurisdiction in which the contract was made, or in which either of the parties is ordinarily resident or headquartered, or in which the contract is to be performed. Best practice is to clearly specify which jurisdiction's laws apply. Parties generally choose the governing law on the basis of their locations, familiarity with legal systems, or the suitability of a jurisdiction's laws.

The complexity of governing law clauses in the context of HPCs merits extensive consideration, and several options have been identified and examined below. All options have no effect on the formal provisions of HPCs, as all HPCs will be almost entirely uniform. Some options would affect the substantive provisions, but all options primarily affects the interpretation of the contracts in the case of a dispute. It is worth considering that the advantages of choosing any particular governing law for its impact on the interpretation of an agreement is likely to be minimal --- despite variations, the ultimate objective of all rules of interpretation is to give effect to the intention of the parties.

\subsubsection{Flexible Approaches}

There are two identifiable approaches which allow users to choose a governing law clause; these have been termed the `flexible' approaches. The first flexible approach is to provide a field into which users are required to enter a jurisdiction of their choice. The second flexible approach is to provide a dropdown list from which users are required to select a jurisdiction. Both approaches have considerable practical drawbacks.

A detrimental side-effect of both approaches outlined is the creation of a gap in enforceability where an arbitrator cannot be found who is familiar with the governing law in question. This would be particularly problematic immediately following the launch of Taskhive, when the number of arbitrators is likely to be small. It would be fairly easy to find arbitrators familiar with common law systems (especially English and American law), and perhaps even Chinese, French, German, Japanese and Russian law (and others), but the likelihood of finding an arbitrator familiar with a relatively obscure law (such as South Sudanese law) would be slim.

\paragraph{Field Entry}

Requiring users to manually enter a jurisdiction would be the most flexible approach. However, this provides significant scope for user error. HPCs are intended to be suitable for users with limited legal knowledge and while some users may have sufficient knowledge to correctly choose an appropriate jurisdiction, it is expected that most will not. This creates significant risk for users and the enforceability of their agreements. Consider the following examples:

\begin{description}
	\item[Scenario 1: `the United Kingdom' is entered.] The United Kingdom is a complex mesh of legal systems. English law and Scottish law are distinct in many aspects, including contract law, as is Northern Irish law to some extent. Devolved legislatures in Northern Ireland, Scotland and Wales have also changed the legal landscape and will continue to do so. `The United Kingdom' is therefore too imprecise for an arbitrator to know which legal principles should apply.
	\item[Scenario 2: `Australia' or `Canada' is entered.] Australia and Canada are federations consisting of states (Australia), provinces (Canada) and territories (both). While they each have unified common law systems, this is subject to statutory modification at the subnational level, producing slight variances between subnational units. The Canadian province of Quebec is a special case also: it has a civil law system descended from French law rather than a common law system derived from English law. Therefore, `Australia' and `Canada' are also too imprecise.
	\item[Scenario 3: `the United States' or `the European Union' is entered.] These are perhaps the vaguest jurisdictions a user would be likely to enter. The US and EU are very similar in constitutional structure: they have central governments with limited powers, but their constituent states and members possess almost all of the law-making authority. The US and EU are significantly more fragmented than Australia and Canada, and an arbitrator would find it impossible to determine which law ought to apply.
\end{description}

\paragraph{Dropdown Entry}

An alternative to field entry is to use a dropdown list. However, compiling such a list would be an herculean task. There are about 210 sovereign states, of which some 30 are federations (such as Australia, Canada, Germany, Russia and the United States), and several others of which have devolved legislatures or separate regional legal systems (such as Belgium, China, France, Spain and the United Kingdom). Potentially there are over 700 separate jurisdictions under whose laws a contract could be made. This is onerous from a development perspective, and cumbersome from a user perspective.

\subsubsection{Rigid Approaches}

Instead of allowing users to choose their own governing law (that is, the flexible approaches), rigid approaches which use non-modifiable clauses would be more beneficial. As with flexible approaches, there are two rigid approaches which would be viable. The first is to include a clause specifying a single governing law for \textit{all} HPCs, and the second is to include clauses containing all the rules and principles of interpretation.

\paragraph{Single Mandatory Governing Law Clause}

Including an identical mandatory governing law clause that specifies the same governing law for every HPC would ensure consistency of interpretation and avoid user errors that could be fatal to the enforceability of an agreement. Choosing English law, Californian law or New York law would likely ensure the availability of arbitrators familiar with the principles of contract interpretation under those laws. However, this brings its own problems:

\begin{itemize}
	\item A single mandatory governing law clause would initially lock out users who may be skilled arbitrators, but are unfamiliar with the specific governing law chosen.
	\item A single mandatory governing law clause would require potential arbitrators to familiarise themselves with the law specified, and to keep their knowledge of that law up-to-date.
	\item A single mandatory governing law clause would introduce the potential for an HPC to be found void on grounds of unlawfulness (also a problem for the flexible approaches).
\end{itemize}

`Unlawfulness' in this context simply means that the agreement has some aspect contrary to law, whether that be that the service provided is prohibited, or the form of payment is prohibited, or a party lacks capacity (such as being a minor) under the law to form a contract, and so on. While it may be desirable from a moral or ethical perspective to discourage illegal behaviour, the reality is that the scope of what is unlawful varies between jurisdictions and over time, often without an obvious moral or ethical basis.

Is it desirable to allow arbitrators to determine that a contract cannot be enforced because it is void? Most likely not given the variations in laws over time and territory can produce unpredictable results as to what is and may be, or stop being, illegal. Specifying a particular jurisdiction in a governing law clause imports such concerns, and is why agreements complex enough to need to be in writing tend to be drafted by legal professionals who are aware of these issues.

\paragraph{Incorporating Principles of Interpretation}

The final and most preferable approach is to incorporate principles of interpretation directly into every HPC. A set of mandatory clauses which set out the rules by which each HPC is to be interpreted has several advantages:

\begin{itemize}
	\item there is no scope for user error, meaning HPCs will be robust and consistent in interpretation,
	\item it allows arbitrators to determine disputes without requiring knowledge of any specific contract law,
	\item it allows users to easily understand how the agreement will be interpreted,
	\item it is jurisdiction-agnostic, removing the possibility of an HPC being found void for illegality and making HPCs immune to changes in the law, and
	\item it can be updated if and when necessary rather than relying on the law to change.
\end{itemize}

English law and its derivatives (including American, Australian and Canadian law) have well-established principles for contract interpretation. Principles for interpreting HPCs can easily be derived from English-based common law rules. Examples of these rules would include that preference be given to literal interpretation where possible; that in cases of ambiguity, reference may be made to other materials such as messages between parties; and that clauses preventing a party from seeking a remedy are invalid.

\section{Dispute Resolution}

Dispute resolution provisions typically include a `forum selection clause' which establishes where a dispute is to adjudicated. This may be broad and allow a number of tribunals to determine a dispute, or it may be narrow and give a specific tribunal exclusive jurisdiction to hear claims. Forum selection clauses cover not only courts, but also arbitral tribunals. Parties may agree to arbitration in accordance with specified rules which are usually promulgated by one of a number of arbitration bodies such as the American Arbitration Association.

HPCs will require a mandatory arbitration clause because the parties will usually be insufficiently identified to be able to bring court proceedings. Arbitration in regards to HPCs will be conducted by a Taskhive user agreed to by the parties in advance. The arbitrator may be selected by the parties based on reputation, cost and expertise, and will interpret the HPC in accordance with interpretation rules included in the agreement. Arbitrators' fees will be determined by negotiation and be incorporated into the agreement.

There are a number of procedural rules which have been professionally drafted and promulgated by various arbitration organisations. Which rules are used is not particularly important, however modifications will be required to adapt any set of rules to be compatible with Taskhive's features. The United Nations Commission on International Trade Law ('UNCITRAL') has published a mature set of arbitration rules, which has many provisions well-suited to online dispute resolution, and is designed specifically with international contracts in mind. However, due to the unique nature of HPCs, substantial modifications would be required to improve communication and overall efficiency. As a result, it would be best to incorporate the arbitration rules directly.

\section{Primary Obligations}

The primary obligations are the core commitments each party makes when executing the contract. They are the `substance' of the transaction --- the work to be performed, and the price to be paid for that work. This is the part of an HPC where users will have the most input. In essence, the primary obligations will consist of two clauses beginning with `The Vendor shall' and `The Vendee shall', followed by their respective obligations which may be broken into subclauses if necessary. There will also be provisions as to time of completion in terms of deliverables and payment. This could be managed by a number of stock options and the ability for users to write their own clauses.

\section{Intellectual Property}

A significant number --- if not a majority --- of HPCs will involve the creation of or dealing with material that attracts intellectual property protection, chiefly copyright (confidential information, which is sometimes considered a form of intellectual property, is considered separately below). There are essentially three ways of dealing with copyright as part of a transaction:

\begin{itemize}
	\item an assignment, in which the copyright owner of a work transfers the ownership of the copyright to another person,
	\item a licence, in which the copyright owner of a work grants another person permission to use the copyright material while retaining ownership, or
	\item no authorship, in which the person dealing with the material does not make a contribution significant enough to attract a copyright of their own.
\end{itemize}

\subsection{Assignment}

An assignment occurs when a copyright owner transfers their ownership to another person. Assignment of copyright is common where the person acquiring ownership of the copyright intends to deal with the material in a way that requires them to be unencumbered. Assignments cannot be revoked and do not require an ongoing relationship between the assignor and assignee (although sometimes and ongoing relationship may be establish, such as the making of royalty payments). Assignments may be beneficial to both parties: the assignee has unfettered use, and the assignor will be able to charge more for an assignment than a licence.

\subsubsection{Partial Assignment}

`Copyright' is a broad term that refers to a bundle of rights relating to a work, which may vary according to its nature. Copyright consists of a number of exclusive rights that only the copyright owner (or owner of a specific exclusive right) may exercise when dealing with the work. It is possible to partially assign copyright by transferring ownership of one or more (but obviously not all) of the exclusive rights, or by transferring ownership in respect of a territory, or by transferring ownership for a limited time, or a combination of these (see the section below on limitations on licences). The original copyright owner will retain the residual rights.

\subsection{Licence}

A licence is used when a copyright owner wishes to permit another person to use a work, but does not want to transfer ownership of the copyright. Licences may be described as exclusive or non-exclusive, and limited or unlimited.

\subsubsection{Exclusive and Non-Exclusive Licences}

An exclusive licence is granted to only one person at a time, and permits the licencee the sole rights to deal with a work in the manner specified. For example: a company may hold an exclusive licence to distribute a video game, meaning that the copyright owner cannot grant anyone else permission to distribute the game while the exclusive licence is valid. Because exclusive licences grant such enormous rights to the licencee, they are often limited by duration, territory or use.

By contrast, non-exclusive licences can be granted to many people at the same time. Non-exclusive licences are common where the use permitted by the licence is valuable to many people regardless of whether they have exclusive rights of use. For example: while a distribution licence is usually significantly more lucrative if it is exclusive (the distributor has no competition), an end-user software licence is likely to be more valuable if it is sold to many people at once (a non-exclusive licence). Non-exclusive licences may be limited by duration or territory, but are most often limited by use. The reason for this is that different uses are likely to vary in their value --- a person who purchases a non-exclusive licence to merely watch a film will pay less than a person who purchases a non-exclusive licence to show the film at a cinema.

\paragraph{Copyleft Licences}

Copyleft licences tend to be an exception to the rule that the more permissive a licence is the more the copyright owner will charge (although copyleft licences are not necessarily `free as in free beer'). `Copyleft' covers a broad range of non-exclusive licences, all of which permit significantly more rights to deal with a work than would be expected from ordinary licences of a comparable cost. It is worth noting that even if a work is licenced under a copyleft licence, this does not prevent a person from charging for that licence, nor does it prevent a person from granting other licences on different terms. For example: a software developer may offer their software under both a copyleft and commercial licence.

Copyleft licences are associated with free and open-source software (`FOSS'; also `FLOSS' for `free/libre and open-source software') and free culture. Open-source licences include, inter alia, the MIT, BSD, GPL, AGPL and LGPL licences, and are predominantly (if not exclusively) used for software code. Creative Commons licences are popular and more suitable licences for non-software literary works, musical works, films, and so on.

Copyleft licences may be suitable where the value of the work to the vendee is not based on the ability to sell further licences. For example: an organisation may need a membership management system developed that allows them to sign up new members, renew existing memberships, and collect annual membership fees. The value of the software to the organisation is in having a usable system for managing members and fees. Whether other people use the system is not a concern, and the organisation may in fact benefit from ongoing development facilitated by the open source nature of the software. If the vendor wishes to release the software under an open-source licence, the vendee probably would not object.

\subsubsection{Limitations on Licences}

Licences --- especially exclusive licences --- are often limited by duration, territory, use, or a combination of these.

A licence limited by duration expires on a specified date (or after a specified time) or when a specified event occurs. For example: the writers of a musicial may authorise a theatre company to perform the musical for a period of up to two weeks, but allow that period to take place within 18 months of executing the contract; the licence expires either at the end of the two week period or the 18 month window, whichever is sooner.

A licence limited by territory is restricted to a specific area or areas; the more clearly defined the better. This is common for distribution, where importers and distributors are granted licences for a whole country or region. For example: a production company sells the streaming rights to a popular programme to one streaming service for the United States, but sells the streaming rights to another service in Australia. This is common where a licencee has not yet entered a market.

A licence limited by use restricts the way a licencee may deal with the licensed work. It is common to allow only one use, because, for example, a film studio will likely only be interested in the film adaptation rights, not publishing or translation rights. An author may licence the native-language publishing rights, translation and foreign-language publishing rights, audio adaptation rights, television adaptation rights, film adaptation rights and stage adaptation rights to various different people.

It is relatively trivial to include such provisions in a contract: these should be a feature available within HPCs, and recommended for exclusive licences.

\subsection{No Authorship}

In some circumstances a vendor will interact with material but will not make a contribution significant enough to be considered a co-author of the work, and will therefore not have a copyright of their own. For example: a person who proofreads an article and corrects the spelling and grammar will not be an author --- even if the person provides ideas, they won't become a co-author unless they express those ideas in tangible form and the expression is incorporated into the work as-published. A simple provision to the effect that the vendor is not an author is sufficient.

The significance of the difference between a vendor-author assigning copyright and a vendor waiving any claim of authorship is in its effect on copyright ownership in the case one party defaults. If the vendor is the author they retain ownership in the event that the vendee defaults. However, if the vendor is not an author, they have no interest in the copyright regardless of whether the contract is performed. Parties will tend more toward assignment when the vendor is being paid to produce a work, and will tend more toward a waiver of authorship when the vendor is dealing with an existing work.

\subsection{Portfolio Exception}

A person who assigns or licences their copyright to another person, or who does not have authorship, may wish to retain (or obtain) the right to use the subject matter in a portfolio of their work. A portfolio exception is a simple statement providing that the author has the right to use or display the work in a portfolio.

\subsection{Flowchart}

\includegraphics[width=1\textwidth]{ipchart.png}

\section{Confidentiality}

Some users may wish to prevent disclosure of the terms of the contract, to protect their reputations in the event that they agree to perform work of a certain nature. For example: a user may agree to work on a project of a religious, political or sexual nature with which they do not want to be publicly associated. Although broad, this is a simple variation of a non-disclosure clause (also called a `confidentiality clause' or `secrecy clause').

Breach of a non-disclosure clause is usually remedied with damages and, if necessary, an injunction. However, because HPCs rely on escrow for awarding damages and breach of non-disclosure provisions may occur years after the contract has been substantially performed, it is impractical (and unfair) to require a party to lock up an amount in escrow for an indefinite period of timet. Similarly, enforcing injunctions via Taskhive is a conceptual challenge. Remedies for breaches of non-disclosure clauses will have to be handled using Taskhive's web-of-trust.

\section{Termination \& Penalties}

When a party breaches, repudiates or wishes to terminate a contract, the other party is usually entitled to damages as compensation, and may in some circumstances have a right to terminate the agreement. Damages may be liquidated, meaning they are specified in the contract itself, or unliquidated, meaning they must be calculated by the court or arbitral tribunal. Liquidated damages are common where the effect of a breach can be quantified in advance with reasonable accuracy, or to avoid calculating unliquidated damages, which can be complex.

A breach of contract does not necessarily give rise to a right of termination. A breach of a term will only give rise to damages; a breach of a condition will give rise to damages and a right of termination. A condition is a \textit{crucial} requirement of the contract, whereas a term is less significant and can be adequately remedied through damages. Consider the following examples:

\begin{itemize}
	\item A hardware store owner hires a person to develop a website to advertise and sell their tools. The website is meant to be finished on a particular date, but the developer finishes it a few days late. The store owner would be entitled to compensation for the delay (by calculating lost income, or using a liquidated damages provision). However, the store owner would not be entitled to terminate the agreement merely because the deadline had passed, because it was not critical that the website be launched on that particular date. The deadline is merely a term and breach is adequately compensated by damages. They would still need to pay the developer, less any damages to which they are entitled.
	\item A pub owner hires a person to design posters advertising St Patrick's Day specials, which they will hang up inside and outside the pub the night before St Patrick's Day. The pub owner needs to allow time for printing and shipping, so the design is needed at least a week before St Patrick's Day. The due date passes without the pub owner receiving the poster design, so it is too late to have the posters printed in time for St Patrick's Day. The pub owner is entitled to damages, and is also entitled to terminate the contract. The pub owner is not required to pay the designer, as it was a condition of the contract that the poster be completed on time.
\end{itemize}

A court will generally not order a vendor to complete their obligations (an order of specific performance), unless the transaction was for the sale and purchase of land. Instead, courts will limit the remedy available to damages and a right of termination (for breach of a condition). Because damages are intended to place the person in the position they would have been if the contract has been performed, the amount of damages will vary according to a number of factors. Consider the following examples:

\begin{itemize}
	\item A person is hired to proofread a 2000 word document at a rate of \$20 per 500 words. They proofread 1500 words and then, for whatever reason, inform the client that they cannot complete the contract. They give the client the 1500 words of proofread material. The proofreader is arguably entitled to \$60 for the work completed (not the full \$80), but the client is arguably entitled to any additional cost incurred in hiring a replacement. If a suitable replacement can be found for \$30 per 500 words, then the proofreader should forfeit \$10 to the client, so that the client is not out of pocket as a result of the proofreader's breach.
	\item A person is hired to compose a jingle for a radio advertisement for a payment of \$500. Having formed a contract, the composer begins work, but soon after the client changes their mind and decides print advertising would be more effective. They wish to cancel the contract with the composer. In this scenario the composer is arguably entitled to the sum of \$500, as it's the amount that they would have received if the client had continued with the contract. Of course, the composer might also agree to termination of the contract without penalty, or the contract may specify a termination fee lower than the full amount.
\end{itemize}

Because HPCs rely on escrow for transactions, awarding damages simply involves subtracting the amount of damages from the contract price paid into escrow by the vendee. For example: if the contract price is \$2000 and the vendee is entitled to \$500 in damages, the vendee pays \$1500 to the vendor and is refunded \$500 in damages. Liquidated damages are preferable to avoid tediously calculating damages. The below table sets out general penalties and the circumstances under which they would apply. Note that `ROT' means `right of termination', and the amount of liquidated damages would vary from contract to contract based on parties' wishes. Also note that in the case of a vendee defaulting on payment, the arbitrator and vendor would authorise the release of funds from escrow, so there is no need for separate penalties.

\begin{tabularx}{\textwidth}{| l | l | X | l |}
\hline
Action 						& Term or condition & Penalty															& ROT   \\
\hline
Vendor terminates & Date is condition & Contract price + liquidated damages & N/A 	\\
Vendor terminates & Date is term			& Liquidated damages									& N/A 	\\
Vendee terminates & N/A 							& Contract price											& N/A 	\\
Vendor defaults		& Date is condition & Contract price + liquidated damages & Yes 	\\
Vendor defaults   & Date is term			& Liquidated damages calculated daily & Maybe \\
\hline
\end{tabularx}

Other conditions may be imposed. For example: if a vendor defaults (does not deliver the work by the agreed date), the vendor is entitled to liquidated damages calculated daily (perhaps \$100/day), but the contract may allow the vendee to terminate once the damages is equal to the contract price.

The above issues are summarised as questions needing answers, listed below. Each will have an answer largely dependent on the individual contract, but all questions need to have answers. This could be done with a list of options from which users can select consequences to events. The answers to the below questions can be used to determine the amounts each party pays into escrow.

\begin{itemize}
	\item What happens if the vendor wishes to terminate?
	\item What happens if the vendee wishes to terminate?
	\item What happens if the vendor does not complete the contract on time?
	\item What happens if the vendee does not complete the contract on time?
\end{itemize}

There are quite a few variables that may be involved with the determination of penalties, so this area should be approached with caution. It should also be remembered that, while these have been called `penalties', they are more like forfeits intended to compensate the non-defaulting party. Courts are generally reluctant to enforce contractual forfeits intended to penalise rather than compensate: the damages due to the non-defaulting party must have a realistic relationship to the loss incurred. For example: a \$15 late payment fee may be acceptable, but a \$2000 late payment fee would likely be excessive.

The clauses in the Contract Template may be sufficient for termination and penalty clauses. The initial release may benefit from having a set of stock clauses that may be modified accordingly.

\section{Contract}

The template below includes all clauses, optional elements and user-selected or -inputted elements. These are indicated thus:

\begin{itemize}
	\item Text in sans-serif typeface is mandatory.
	\item Text in monospaced typeface indicates user interaction.
	\item Capitalised text in monospaced typeface with a yellow background (\texttt{\colorbox{branding}{EXAMPLE}}) is mandatory but must be inputted by the user. If included in an optional clause, it is mandatory \textit{if} that clause is selected.
	\item Text enclosed in square brackets (\texttt{[} and \texttt{]}) indicates a user choice. These may be nested: \texttt{[optional clause [optional subclause]]}. There may be several optional nested subclauses, any number of which may be selected by a user.
	\item Text enclosed in curly braces (\texttt{\{} and \texttt{\}}) indicates the user \textit{must} choose one clause from a set of options: \texttt{\{[exclusive clause \#1] [exclusive clause \#2 [optional subclause \#2A]]\}}. In this example the user must choose either exclusive clause \#1 or exclusive clause \#2, and if the latter they may also choose optional subclause \#2A.
\end{itemize}

\subsection{HiveProtected Master Contract}

\raggedright

\textbf{Parties}

\begin{enumerate}
	\item This agreement is between:
	\begin{enumerate}
		\item \texttt{\colorbox{branding}{VENDOR'S PUBLIC KEY}}, address \texttt{\colorbox{branding}{VENDOR'S ELECTRONIC ADDRESS}} (hereafter `the Vendor'), and
		\item \texttt{\colorbox{branding}{VENDEE'S PUBLIC KEY}}, address \texttt{\colorbox{branding}{VENDEE'S ELECTRONIC ADDRESS}} (hereafter `the Vendee').
	\end{enumerate}
\end{enumerate}

\textbf{Interpretation}

\begin{enumerate}[resume]
	\item In this contract:\footnote{Most of these definitions are repeated or stated later in the contract. They are included here because they are terms that are used regularly, and it is helpful to have a single section where common terms are explicitly defined.}
	\begin{enumerate}
		\item `the Arbitrator' means the person identified in Clause (5),
		\item `the Claimant' means, in respect of arbitral proceedings, the party issuing a statement of claim,
		\item `contract price' means the total sum payable by the Vendee to the Vendor on performance of this agreement if this agreement was to be performed on or before the date of completion,\footnote{This is the amount that the Vendee agrees to pay the Vendee, but does not include any fees, penalties or compensational amounts. It's what they agreed the work was worth.}
		\item `contract term' means the number of days between and including the date of execution and the date of completion,\footnote{If the date of execution was 5 July and the date of completion was 30 July, the contract term would be 26 days.}
		\item `date of completion' means the date on which the obligations under this agreement are to have been performed,
		\item `date of execution' means the date on which this agreement was entered into,
		\item `the Respondent' means, in respect of arbitral proceedings, the party against whom a claim is made,
		\item `the Vendee' means the party identified in Clause (1)(b), and
		\item `the Vendor' means the party identified in Clause (1)(a).
	\end{enumerate}
	\item This contract shall be interpreted in accordance with the following rules:\footnote{These rules will be derived from English law.}
\end{enumerate}

\textbf{Dispute Resolution}

\begin{enumerate}[resume]
	\item Any dispute, controversy or claim arising out of or relating to this contract, or the breach, termination or invalidity thereof, shall be settled by arbitration.
	\item The Arbitrator shall be \texttt{\colorbox{branding}{ARBITRATOR'S PUBLIC KEY}}, address \texttt{\colorbox{branding}{ARBITRATORS'S ELECTRONIC ADDRESS}} (hereafter `the Arbitrator').
	\item The language to be used in the arbitral proceedings shall be \texttt{\colorbox{branding}{LANGUAGE}}.\footnote{The parties can choose the language.}
	\item Arbitral proceedings shall be conducted in accordance with the following rules:
	\begin{enumerate}
		\item All communications between the parties and the Arbitrator must be sent and received at their respective electronic addresses listed in Clauses (1) and (5).
		\item Periods of time shall be calculated by reference to Universal Time Coordinated, shall not include Saturdays or Sundays, and shall begin on the day following the day on which a statement of claim or statement of defence is sent.\footnote{Time will be calculated according to GMT+0 and only include weekdays. Because people will be in different timezones, it is easier to use a single reference point. Public holidays are ignored because they vary from place to place.}
		\item The party initiating proceedings (hereafter `the Claimant') shall communicate to the other party (hereafter `the Respondent') and the Arbitrator a statement of claim setting out the alleged factual and contractual basis of the claim, the amount involved (if any), and the relief or remedy sought.
		\item Within \texttt{\colorbox{branding}{NUMBER}} days of receiving the statement of claim, the Respondent shall communicate to the Claimant and the Arbitrator a statement of defence setting out its response to the statement of claim, and including any counterclaims, and any amounts, relief or remedies sought.\footnote{Users can choose how much time they want to allow for a response to the statement of claim.}
		\item Each party may amend or supplement its statement of claim or statement of defence at any point during the proceedings unless the Arbitrator considers it inappropriate on the basis that it would cause undue prejudice to the other party.
		\item Subject to these rules, the Arbitrator shall conduct proceedings in whatever manner the Arbitrator considers appropriate, but must:
		\begin{enumerate}
			\item treat each party equally and ensure each party has a reasonable opportunity of presenting its case,
			\item conduct proceedings fairly, efficiently and without unnecessary delay, and
			\item respect the right of each party to protect their identity.
		\end{enumerate}
		\item The Arbitrator shall endeavour to have the proceedings concluded and to issue their determination:
		\begin{enumerate}
			\item within \texttt{\colorbox{branding}{NUMBER}} days of receiving the statement of defence, or\footnote{Users can choose how much time they want to allow for an arbitrator to resolve a dispute.}
			\item if the Respondent does not provide a statement of defence, within \texttt{\colorbox{branding}{NUMBER}} days of receiving the statement of claim.\footnote{This should be required by technical means to be greater than the amount of time that the Respondent has to respond.}
		\end{enumerate}
		\item The Arbitrator shall determine liability for the Arbitrator's fees.\footnote{Awarding costs, essentially.}
	\end{enumerate}
\end{enumerate}

\textbf{Primary Obligations}\footnote{These must be filled in by the users.}

\begin{enumerate}[resume]
	\item The Vendor shall \ldots{}
	\item The Vendee shall \ldots{}
\end{enumerate}

\textbf{Intellectual Property}

\begin{enumerate}[resume]
	\item \texttt{\{[Upon completion of this contract, the Vendor \{[assigns to the Vendee] [grants to the Vendee \{[an exclusive] [a non-exclusive} \texttt{[\colorbox{branding}{COPYLEFT LICENCE}]]\} licence with regard to]\} all copyrights that subsist in works created for and delivered to the Vendee in performance of Clause (8) [for a period of \colorbox{branding}{DURATION}] [for the territory (or territories) of \colorbox{branding}{TERRITORY}] [for the purpose or purposes of \colorbox{branding}{USE}]] [The Parties agree that copyright does not subsist in any derivative works created for and delivered to the Vendee in performance of Clause (8)]\}} \texttt{[but the Vendor retains and is granted by the Vendee all rights necessary to include the works and parts thereof in a portfolio of the Vendor's works]]}.\footnote{Very complicated range of options, but technically straightforward.}
\end{enumerate}

\textbf{Termination \& Penalties}

\begin{enumerate}[resume]
	\item The Vendor or Vendee may terminate this agreement by giving \texttt{\colorbox{branding}{PERIOD OF DAYS}} notice.\footnote{Users can set the amount of notice required (eg, one day, two days, etc).}
	\item Unless otherwise agreed:
	\begin{enumerate}
			\item if the Vendee terminates this agreement, the Vendor shall be entitled to \texttt{\{[the full contract price] [[\texttt{\colorbox{branding}{FIXED AMOUNT}}] [a proportion of the contract price calculated by dividing the contract price by the number of days of the contract term multiplied by the number of days accrued since the date of execution] with the remainder to be returned to the Vendee]},\footnote{This covers the effect of termination by the Vendee (client). Users can select whether the Vendor (seller) gets the entire contract price, or a fixed amount, or an amount calculated on a pro rata basis. If a fixed amount or pro rata amount, the remainder is to be returned to the Vendee. The fixed amount should therefore be required to be less than the contract price.}
			\item if the Vendor terminates this agreement, the Vendee shall be entitled to reclaim the whole of the contract price \texttt{[and \colorbox{branding}{AMOUNT} per day accrued from the date of execution [up to \colorbox{branding}{TOTAL AMOUNT}]]},\footnote{This covers the effect of termination by the Vendor (seller). The Vendee (client) is entitled to the whole of the contract price, and users may select whether they should also be entitled to compensation for each day elapsed, and if appropriate a maximum amount (eg, \$100 per day up to \$700; even if the termination was 10 days after the date of execution, the compensation would be capped at \$700).}
			\item if the Vendor does not perform this agreement by the date of completion, the Vendee shall be entitled to:
			\begin{enumerate}
					\item terminate the agreement at any time beyond the date of completion unless the Vendee has performed the agreement, and\footnote{Allows the Vendee to terminate the agreement if the contract is not performed on time.}
					\item deduct and reclaim \texttt{\colorbox{branding}{AMOUNT}} (or part thereof) from the contract price per day in excess of the date of completion until the contract price is \texttt{\colorbox{branding}{0 CURRENCY}}, and\footnote{Allows the Vendee to reduce the contract price by a specified amount each day after the date of completion until it reaches nil.}
			\end{enumerate}
			\item if the contract price has been wholly deducted and reclaimed in accordance with Clause (12)(c)(ii), the contract shall be terminated immediately \texttt{[and the Vendor shall be entitled to damages of \colorbox{branding}{AMOUNT}]}.\footnote{If the contract price is deducted and becomes nil, the contract is terminated and users may opt to permit the Vendor to claim specified damages.}
		\end{enumerate}
\end{enumerate}

\textbf{Escrow}

\begin{enumerate}[resume]
	\item Within \texttt{\colorbox{branding}{NUMBER OF DAYS}} of the date of execution:
	\begin{enumerate}
		\item the Vendee shall pay \texttt{\colorbox{branding}{AMOUNT}} to be held in escrow, and
		\item the Vendor shall pay \texttt{\colorbox{branding}{AMOUNT}} to be held in escrow.
	\end{enumerate}
	\item If either party defaults on its obligation to pay into escrow the amount specified within the time specified:\footnote{Allows the contract to be cancelled without penalty.}
	\begin{enumerate}
		\item the other party shall be entitled to terminate the agreement until the full amount is paid into escrow,
		\item each party shall be entitled to reclaim in full any amounts already paid into escrow, and
		\item neither party shall be penalised or owe compensation to another party.
	\end{enumerate}
\end{enumerate}


\texttt{Confidentiality}

\begin{enumerate}[resume]
	\item \texttt{The Vendor, the Vendee and the Arbitrator shall not disclose the terms of this contract to any person who is not a party to or the Arbitrator of this contract}.
\end{enumerate}

\end{document}
